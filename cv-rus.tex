% Workaround for problem  
% Command \CYRE unavailable in encoding OT1.
% http://tex.stackexchange.com/questions/86078/strange-latex-compilation-errors-triggered-by-the-number-of-lines-on-the-page


\documentclass[10pt]{article}
\usepackage{array, xcolor}
\usepackage[margin=3cm]{geometry}
\usepackage{hyperref}

\usepackage{fontspec}
\defaultfontfeatures{Scale=MatchLowercase,Mapping=tex-text}
\setmainfont[Ligatures=TeX]{DejaVu Serif}
\setsansfont[Ligatures=TeX]{DejaVu Sans}
\setmonofont{DejaVu Sans Mono}

\usepackage[utf8]{inputenc}
\usepackage[russian,english]{babel}
 
\title{\bfseries\Huge Олег Шпынов}
\author{oleg.shpynov@gmail.com}
\date{}
 
\definecolor{lightgray}{gray}{0.8}
\newcolumntype{L}{>{\raggedleft}p{0.14\textwidth}}
\newcolumntype{R}{p{0.8\textwidth}}
\newcommand\VRule{\color{lightgray}\vrule width 0.5pt}
 
\begin{document}
\maketitle
\vspace{1em}
\begin{minipage}[ht]{0.75\textwidth}
Шуваловский пр. 84 135\\
Санкт-Петербург\\
Россия\\
197345
\end{minipage}
\begin{minipage}[ht]{0.25\textwidth}
7 Апреля 1986\\
+7 904 331 35 04
\end{minipage}
\vspace{20pt}
 
\section*{Общее}
Сильная математическа подготовка. Более 9 лет опытка разработки программных продуктов.\\
Опыт в сферах биоинформатики, машинного обучения, статистики, компиляторов. \\
Хорошие коммуникативные и навыки командной работы.
 
\section*{Опыт работы}
\begin{tabular}{L!{\VRule}R}
2012-- & {\bf Исследователь JetBrains Biolabs}\\
& Целями \href{http://research.jetbrains.org/groups/biolabs}{научной группы JetBrains Biolabs} являются исследования фундаментальных механизмов, определящих процессы эпигенетической регуляции у человека и других животных, и роль этих механизмов в процессах дифференциации и старения клеток.\\
& Координация деятельности проекта, анализ NGS данных, генерация семантических правил, разработка пакета анализа эпигенетических данных, серверный геномный браузер, Байесовские вероятностные сети.\\
& \\
2012--2014 & {\bf Научный руководитель}\\
& \href{http://bioinformaticsinstitute.ru/teachers/shpynov}{Научный руководитель} двух успешно защитившихся магистров по направлению "Теоретическая Биоинформатика" Санкт-Петербургского Академического Университета.\\
& \\
2012--2013 & {\bf Технический руководитель}\\
& Руководство технической частью, архитектура и дизайн студенческого проекта Genome Query, серверного программного продукта для поиска геномных выравниваний, генов, итд с помощью удобного синтаксиса запросов. \\
& \\
2006--2013 & {\bf Старший разработчкик JetBrains}\\
& Основатель среды разработки \href{http://jetbrains.com/ruby}{RubyMine IDE}. Разработка архитектуры для поддержки языка программрования Ruby в интегрированной среде разработки IntellIJ IDEA. Разработка VIM эмулятора для IntelliJ IDE. Интеграция сервиса контроля версий и хранения исходных кодов Github. \\
& Низкоуровневая поддержка языков, разработка парсеров, типовых систем, статического анализа, итд. \\
\end{tabular}
 
 
\section*{Образование}
\begin{tabular}{L!{\VRule}R}
2015 & Школа по машинному обучению и искуственному интеллекту, Microsoft, Россия \\
& Машинное обучение, искуственный интеллект, статистика. \\ 
& \\
2011--2012 & Санкт-Петербургский Академический Университет — научно-образовательный центр нанотехнололгий РАН, Россия\\
& Биоинформатика, молекулярная биология, статистика. \\
& \\
2003--2008 & {\bf Санкт-Петербургский Государственный Университет, Россия}\\[5pt]
& Математико-механический факультет, кафедра Системного программирования. \\
& \\
2008--2010 & Санкт-Петербургский Государственный Университет, Россия \\
& Аспирант Математико-механический факультета, кафедры Системного программирования. \\
\end{tabular}

\section*{Публикации}
\begin{tabular}{L!{\VRule}R}
2015 & S. Lebedev, R. Chernyatchik, O. Shpynov "CMeth: a Bayesian semiparametric model
for differential methylation analysis", препринт доступен по \href{http://bit.ly/cmeth-preprint}{ссылке}
\end{tabular}

 
\section*{Языки}
\begin{tabular}{L!{\VRule}R}
Русский & Родной\\
Английский & Свободный\\
Французский & Начинающий \\
Немецкий & Начинающий \\
\end{tabular}



\section*{Достижения}
\begin{tabular}{L R}
& Закончил с отличием Санкт-Петербургский Государственный Университет\\
& \\
& Вошел в топ 20 из 500+ комманд в соревнованию по функциональному программирования ICFPC 2012\\
& \\
& Выступал в комманде университета в региональных чемпионатах по программированию ACM \\
& \\
&  Неоднократный победитель школьных олимпиад Сантк-Петербуга по математике, физике и программированию
\end{tabular}

\section*{Увлечения}
\begin{tabular}{L R}
	& Биоинформатика, машинное обучение, искусственный интеллект, компиляторы, экономика.\\
	& Путешествия, походы, сноуборд, велосипед, дайвинг, фотография.\\
\end{tabular}
 
\end{document}